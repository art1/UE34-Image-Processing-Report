%%%%%%%%%%%%%%%%% TASK 1 %%%
\section{Introduction - Image Processing for Earth Observation}
Image processing and the analysis of image data plays an important role especially for Earth Observations, but also in other fields related to space applications that make use of images, e.g. optical navigation or optical attitude control systems. In this report, we present a thorough overview of examples for image processing techniques, especially applying examples in Matlab. We focus on colour spaces for images, fourier transforms of basic figures, erosion and delatation and classification. 

\subsection{Computer Vision}
From a mathematical point of view,an Image is an application that associates a value $I(u,v)$ to a pixel $(u,v)$. Pixels are atomic elements arranged in rows and columns,  such that the size of an image is given by the number of rows and the number of columns. Therefore, an image can be processed by using a matrix representation, where each matrix element $(u,v)$  represents a pixel containing the value $I(u,v)$ of the pixel.. In the case of multispectral images, i.e. colour images like RGB images (see following chapter), an image is an application that associates saveral values, e.g. $I_{1}(u,v)$ $I_{2}(u,v)$ and $I_{3}(u,v)$ to a pixel $(u,v)$, where the values $I_{i}$ represent a different colour like red, green and blue in RGB images. Therefore, each matrix element $(u,v)$ contains a vector of values. This matrix notation is the basic of image processing using Matlab.

\section{Greyscale and Colour images}



 
\subsection{Filtering}

\subsection{Image Processing \& Analysis}

\subsection{Pattern Recognition}

\subsection{Geometry}


%%%%%%%%%%%%%%%%% TASK 3 %%%
\section{Computer Vision \& Morphology}

\subsection{Erosion \& Dilatation}

\subsection{Morphological Filtering}

\subsection{Morphological Skeletonization \& Segmentation}


%%%%%%%%%%%%%%%%% TASK 3 %%%
\section{Data Analysis \& Processing}
After having extensively discussed Image Processing and Image Analysis in the previous chapters we will now move on to Data Analysis. \\In general, Data Analysis is an extension or generalisation of Image Analysis - in particular, an image can also be seen as a set of data with a specific topology. However, data analysis is not limited to images but can be used for a vast amount of applications, be it business, science or others, where a set of data is given and information has to be extracted. Ultimately, this is the goal of data analysis: shaping, modelling and analysing the data to gain information or conclusions from given data.
Since this course puts an emphasis on analysing and processing images, all the following examples will be performed on and explained by using images as a data set.

The term \textit{data} itself can be interpreted and defined in multiple ways, which raises the need for a definition in our context of data analysis of images.
In general data can be considered as an element taken from a set of data-elements. Each data-element can then be seen as a set of different components, attributes, parameters etc. - defining what the data-element is composed of - and are often called \textit{descriptors}. Mathematically speaking, this means that our data $\boldsymbol{x}$ is associated to a vector in $\boldsymbol{\mathbb{R}^n}$ containing the descriptors,
\begin{equation*}
	x = (c_1 ... c_n)^T
\end{equation*}
which is also called the \textit{state space E} of our data.
These descriptors are the crucial part when analysing data, since they are defining a set of rules according to which we analyse the raw data.

If this terminology is applied to images, the pixels or a range of pixels in each image can be seen as such an descriptor, thus the image itself as our data-element.

\subsection{Classification}
\begin{figure}[h!]
	\centering
	\includegraphics[width=\textwidth/2-5em]{images/data_analysis.png}
	\caption{Steps for classification \protect\footnotemark}
	\label{fig:data_anal}
\end{figure}
\footnotetext{source: lecture notes by Emmanuel Zenou}
Having a data set that consists of descriptors essentially allows us to \textit{label} this particular data set. This labelling of data is usually also described as \textit{Classification}, meaning that each data-element or sub-data-element is assigned a particular \textit{class}. In image analysis for space applications typically the pixels in an image have to be classified.

In general, classification of data consists of three steps,
\begin{enumerate}
	\item Formatting
	\item Modelling
	\item Classification
\end{enumerate}
whereas these steps are not compulsory. The formatting step usually consists of finding good descriptors, changing the state space (e.g. by re-shaping), pre-processing data (e.g. filtering) and so on.\\
The modelling step requires to find a model and its optimal parameters to fit the data, but also to fit the model output to the data and to validate the model.
In the final Classification part, the task is to find classes and a classification rule according to which distinct data will be labeled.

There are two main forms of classification, \textit{Supervised} and \textit{Unsupervised}, of which both are described and explained in detail in the following chapters.


\subsection{Supervised Classification}

\subsection{Unsupervised Classification}
